%% Generated by Sphinx.
\def\sphinxdocclass{report}
\documentclass[letterpaper,10pt,portuges]{sphinxmanual}
\ifdefined\pdfpxdimen
   \let\sphinxpxdimen\pdfpxdimen\else\newdimen\sphinxpxdimen
\fi \sphinxpxdimen=.75bp\relax
\ifdefined\pdfimageresolution
    \pdfimageresolution= \numexpr \dimexpr1in\relax/\sphinxpxdimen\relax
\fi
%% let collapsible pdf bookmarks panel have high depth per default
\PassOptionsToPackage{bookmarksdepth=5}{hyperref}

\PassOptionsToPackage{booktabs}{sphinx}
\PassOptionsToPackage{colorrows}{sphinx}

\PassOptionsToPackage{warn}{textcomp}
\usepackage[utf8]{inputenc}
\ifdefined\DeclareUnicodeCharacter
% support both utf8 and utf8x syntaxes
  \ifdefined\DeclareUnicodeCharacterAsOptional
    \def\sphinxDUC#1{\DeclareUnicodeCharacter{"#1}}
  \else
    \let\sphinxDUC\DeclareUnicodeCharacter
  \fi
  \sphinxDUC{00A0}{\nobreakspace}
  \sphinxDUC{2500}{\sphinxunichar{2500}}
  \sphinxDUC{2502}{\sphinxunichar{2502}}
  \sphinxDUC{2514}{\sphinxunichar{2514}}
  \sphinxDUC{251C}{\sphinxunichar{251C}}
  \sphinxDUC{2572}{\textbackslash}
\fi
\usepackage{cmap}
\usepackage[T1]{fontenc}
\usepackage{amsmath,amssymb,amstext}
\usepackage{babel}



\usepackage{tgtermes}
\usepackage{tgheros}
\renewcommand{\ttdefault}{txtt}



\usepackage[Sonny]{fncychap}
\ChNameVar{\Large\normalfont\sffamily}
\ChTitleVar{\Large\normalfont\sffamily}
\usepackage{sphinx}

\fvset{fontsize=auto}
\usepackage{geometry}


% Include hyperref last.
\usepackage{hyperref}
% Fix anchor placement for figures with captions.
\usepackage{hypcap}% it must be loaded after hyperref.
% Set up styles of URL: it should be placed after hyperref.
\urlstyle{same}

\addto\captionsportuges{\renewcommand{\contentsname}{Contents:}}

\usepackage{sphinxmessages}
\setcounter{tocdepth}{3}
\setcounter{secnumdepth}{3}


\title{Exemplos Diversos}
\date{02 abr., 2025}
\release{}
\author{Pedro Guerreiro}
\newcommand{\sphinxlogo}{\vbox{}}
\renewcommand{\releasename}{}
\makeindex
\begin{document}

\ifdefined\shorthandoff
  \ifnum\catcode`\=\string=\active\shorthandoff{=}\fi
  \ifnum\catcode`\"=\active\shorthandoff{"}\fi
\fi

\pagestyle{empty}
\sphinxmaketitle
\pagestyle{plain}
\sphinxtableofcontents
\pagestyle{normal}
\phantomsection\label{\detokenize{index::doc}}


\sphinxAtStartPar
Add your content using \sphinxcode{\sphinxupquote{reStructuredText}} syntax. See the
\sphinxhref{https://www.sphinx-doc.org/en/master/usage/restructuredtext/index.html}{reStructuredText}
documentation for details.

\sphinxstepscope


\chapter{main module}
\label{\detokenize{main:module-main}}\label{\detokenize{main:main-module}}\label{\detokenize{main::doc}}\index{module@\spxentry{module}!main@\spxentry{main}}\index{main@\spxentry{main}!module@\spxentry{module}}

\section{Utilitário de Operações Matemáticas, Texto e Listas}
\label{\detokenize{main:utilitario-de-operacoes-matematicas-texto-e-listas}}
\sphinxAtStartPar
Este módulo contém o programa principal que oferece um menu interativo
para executar diversas operações em matemática, manipulação de texto e
operações com listas.

\sphinxAtStartPar
O programa apresenta um menu principal e submenus que permitem ao
utilizador selecionar diferentes funcionalidades implementadas nos
módulos importados.

\begin{sphinxadmonition}{note}{Nota:}
\sphinxAtStartPar
Este programa é um exemplo de como organizar um código Python
em modulos, permitindo a reutilização de funções em diferentes
contextos.
\end{sphinxadmonition}

\begin{sphinxadmonition}{warning}{Aviso:}
\sphinxAtStartPar
Este código é apenas um exemplo e não deve ser utilizado em
produção sem as devidas validações e tratamento de exceções.
\end{sphinxadmonition}
\begin{quote}\begin{description}
\sphinxlineitem{Autor}
\sphinxAtStartPar
Pedro Guerreiro

\sphinxlineitem{Data}
\sphinxAtStartPar
1 de abril de 2025

\sphinxlineitem{Versão}
\sphinxAtStartPar
1.0

\end{description}\end{quote}
\index{main() (no módulo main)@\spxentry{main()}\spxextra{no módulo main}}

\begin{fulllineitems}
\phantomsection\label{\detokenize{main:main.main}}
\pysigstartsignatures
\pysiglinewithargsret
{\sphinxcode{\sphinxupquote{main.}}\sphinxbfcode{\sphinxupquote{main}}}
{}
{}
\pysigstopsignatures
\sphinxAtStartPar
Função principal que controla o fluxo do programa.

\sphinxAtStartPar
Esta função apresenta o menu principal e direciona para os submenus
baseando\sphinxhyphen{}se na escolha do utilizador. Executa as operações
solicitadas chamando as funções dos módulos importados com os
parâmetros apropriados.
\begin{quote}\begin{description}
\sphinxlineitem{Retorno}
\sphinxAtStartPar
None

\sphinxlineitem{Tipo de retorno}
\sphinxAtStartPar
NoneType

\end{description}\end{quote}

\sphinxAtStartPar
Por exemplo:

\begin{sphinxVerbatim}[commandchars=\\\{\}]
\PYG{g+gp}{\PYGZgt{}\PYGZgt{}\PYGZgt{} }\PYG{n}{main}\PYG{p}{(}\PYG{p}{)}
\PYG{g+go}{\PYGZsh{} Inicia o programa interativo com menus}
\end{sphinxVerbatim}

\end{fulllineitems}

\index{menu() (no módulo main)@\spxentry{menu()}\spxextra{no módulo main}}

\begin{fulllineitems}
\phantomsection\label{\detokenize{main:main.menu}}
\pysigstartsignatures
\pysiglinewithargsret
{\sphinxcode{\sphinxupquote{main.}}\sphinxbfcode{\sphinxupquote{menu}}}
{}
{}
\pysigstopsignatures
\sphinxAtStartPar
Apresenta o menu principal e processa a escolha do utilizador.

\sphinxAtStartPar
A função exibe as opções disponíveis e valida se a escolha está
dentro do intervalo permitido.
\begin{quote}\begin{description}
\sphinxlineitem{Retorno}
\sphinxAtStartPar
Número inteiro correspondente à opção escolhida

\sphinxlineitem{Tipo de retorno}
\sphinxAtStartPar
int

\end{description}\end{quote}

\sphinxAtStartPar
Por exemplo:

\begin{sphinxVerbatim}[commandchars=\\\{\}]
\PYG{g+gp}{\PYGZgt{}\PYGZgt{}\PYGZgt{} }\PYG{n}{menu}\PYG{p}{(}\PYG{p}{)}
\PYG{g+go}{Escolha uma opção:}
\PYG{g+go}{1. Operações Matemáticas}
\PYG{g+go}{2. Operações com Texto}
\PYG{g+go}{3. Operações com Listas}
\PYG{g+go}{4. Sair}
\PYG{g+go}{Digite a opção desejada: 1}
\PYG{g+go}{1}
\end{sphinxVerbatim}

\end{fulllineitems}

\index{menu\_lista() (no módulo main)@\spxentry{menu\_lista()}\spxextra{no módulo main}}

\begin{fulllineitems}
\phantomsection\label{\detokenize{main:main.menu_lista}}
\pysigstartsignatures
\pysiglinewithargsret
{\sphinxcode{\sphinxupquote{main.}}\sphinxbfcode{\sphinxupquote{menu\_lista}}}
{}
{}
\pysigstopsignatures
\sphinxAtStartPar
Apresenta o submenu de operações com listas e processa a escolha.

\sphinxAtStartPar
A função exibe as operações de lista disponíveis e valida se
a escolha está dentro do intervalo permitido.
\begin{quote}\begin{description}
\sphinxlineitem{Retorno}
\sphinxAtStartPar
Número inteiro correspondente à opção escolhida

\sphinxlineitem{Tipo de retorno}
\sphinxAtStartPar
int

\end{description}\end{quote}

\sphinxAtStartPar
Por exemplo:

\begin{sphinxVerbatim}[commandchars=\\\{\}]
\PYG{g+gp}{\PYGZgt{}\PYGZgt{}\PYGZgt{} }\PYG{n}{menu\PYGZus{}lista}\PYG{p}{(}\PYG{p}{)}
\PYG{g+go}{Operações com Listas:}
\PYG{g+go}{1. Filtrar Pares}
\PYG{g+go}{2. Calcular Frequência}
\PYG{g+go}{3. Agrupar por Tamanho}
\PYG{g+go}{Digite a opção desejada: 1}
\PYG{g+go}{1}
\end{sphinxVerbatim}

\end{fulllineitems}

\index{menu\_matematica() (no módulo main)@\spxentry{menu\_matematica()}\spxextra{no módulo main}}

\begin{fulllineitems}
\phantomsection\label{\detokenize{main:main.menu_matematica}}
\pysigstartsignatures
\pysiglinewithargsret
{\sphinxcode{\sphinxupquote{main.}}\sphinxbfcode{\sphinxupquote{menu\_matematica}}}
{}
{}
\pysigstopsignatures
\sphinxAtStartPar
Apresenta o submenu de operações matemáticas e processa a escolha.

\sphinxAtStartPar
A função exibe as operações matemáticas disponíveis e valida se
a escolha está dentro do intervalo permitido.
\begin{quote}\begin{description}
\sphinxlineitem{Retorno}
\sphinxAtStartPar
Número inteiro correspondente à opção escolhida

\sphinxlineitem{Tipo de retorno}
\sphinxAtStartPar
int

\end{description}\end{quote}

\sphinxAtStartPar
Por exemplo:

\begin{sphinxVerbatim}[commandchars=\\\{\}]
\PYG{g+gp}{\PYGZgt{}\PYGZgt{}\PYGZgt{} }\PYG{n}{menu\PYGZus{}matematica}\PYG{p}{(}\PYG{p}{)}
\PYG{g+go}{Operações Matemáticas:}
\PYG{g+go}{1. Média Ponderada}
\PYG{g+go}{2. Desvio Padrão}
\PYG{g+go}{3. Conversão de Temperatura}
\PYG{g+go}{Digite a opção desejada: 1}
\PYG{g+go}{1}
\end{sphinxVerbatim}

\end{fulllineitems}

\index{menu\_texto() (no módulo main)@\spxentry{menu\_texto()}\spxextra{no módulo main}}

\begin{fulllineitems}
\phantomsection\label{\detokenize{main:main.menu_texto}}
\pysigstartsignatures
\pysiglinewithargsret
{\sphinxcode{\sphinxupquote{main.}}\sphinxbfcode{\sphinxupquote{menu\_texto}}}
{}
{}
\pysigstopsignatures
\sphinxAtStartPar
Apresenta o submenu de operações com texto e processa a escolha.

\sphinxAtStartPar
A função exibe as operações de texto disponíveis e valida se
a escolha está dentro do intervalo permitido.
\begin{quote}\begin{description}
\sphinxlineitem{Retorno}
\sphinxAtStartPar
Número inteiro correspondente à opção escolhida

\sphinxlineitem{Tipo de retorno}
\sphinxAtStartPar
int

\end{description}\end{quote}

\sphinxAtStartPar
Por exemplo:

\begin{sphinxVerbatim}[commandchars=\\\{\}]
\PYG{g+gp}{\PYGZgt{}\PYGZgt{}\PYGZgt{} }\PYG{n}{menu\PYGZus{}texto}\PYG{p}{(}\PYG{p}{)}
\PYG{g+go}{Operações com Texto:}
\PYG{g+go}{1. Contar Vogais}
\PYG{g+go}{2. Gerar Acrônimo}
\PYG{g+go}{3. Criptografia César}
\PYG{g+go}{Digite a opção desejada: 2}
\PYG{g+go}{2}
\end{sphinxVerbatim}

\end{fulllineitems}

\index{pedir\_lista() (no módulo main)@\spxentry{pedir\_lista()}\spxextra{no módulo main}}

\begin{fulllineitems}
\phantomsection\label{\detokenize{main:main.pedir_lista}}
\pysigstartsignatures
\pysiglinewithargsret
{\sphinxcode{\sphinxupquote{main.}}\sphinxbfcode{\sphinxupquote{pedir\_lista}}}
{\sphinxparam{\DUrole{n}{mensagem}}}
{}
\pysigstopsignatures
\sphinxAtStartPar
Solicita ao utilizador uma lista de strings.

\sphinxAtStartPar
A função apresenta uma mensagem ao utilizador e espera que este
insira uma sequência de valores separados por vírgula.
\begin{quote}\begin{description}
\sphinxlineitem{Parâmetros}
\sphinxAtStartPar
\sphinxstyleliteralstrong{\sphinxupquote{mensagem}} (\sphinxstyleliteralemphasis{\sphinxupquote{str}}) \textendash{} Texto a apresentar ao utilizador para solicitar a entrada

\sphinxlineitem{Retorno}
\sphinxAtStartPar
Lista de strings inseridas pelo utilizador

\sphinxlineitem{Tipo de retorno}
\sphinxAtStartPar
list{[}str{]}

\end{description}\end{quote}

\sphinxAtStartPar
Por exemplo:

\begin{sphinxVerbatim}[commandchars=\\\{\}]
\PYG{g+gp}{\PYGZgt{}\PYGZgt{}\PYGZgt{} }\PYG{n}{pedir\PYGZus{}lista}\PYG{p}{(}\PYG{l+s+s2}{\PYGZdq{}}\PYG{l+s+s2}{Digite palavras: }\PYG{l+s+s2}{\PYGZdq{}}\PYG{p}{)}
\PYG{g+go}{Digite palavras: olá,mundo,python}
\PYG{g+go}{[\PYGZsq{}olá\PYGZsq{}, \PYGZsq{}mundo\PYGZsq{}, \PYGZsq{}python\PYGZsq{}]}
\end{sphinxVerbatim}

\end{fulllineitems}

\index{pedir\_lista\_floats() (no módulo main)@\spxentry{pedir\_lista\_floats()}\spxextra{no módulo main}}

\begin{fulllineitems}
\phantomsection\label{\detokenize{main:main.pedir_lista_floats}}
\pysigstartsignatures
\pysiglinewithargsret
{\sphinxcode{\sphinxupquote{main.}}\sphinxbfcode{\sphinxupquote{pedir\_lista\_floats}}}
{\sphinxparam{\DUrole{n}{mensagem}}}
{}
\pysigstopsignatures
\sphinxAtStartPar
Solicita ao utilizador uma lista de números decimais.

\sphinxAtStartPar
A função apresenta uma mensagem ao utilizador e espera que este
insira uma sequência de números separados por vírgula. Cada valor
é convertido para float.
\begin{quote}\begin{description}
\sphinxlineitem{Parâmetros}
\sphinxAtStartPar
\sphinxstyleliteralstrong{\sphinxupquote{mensagem}} (\sphinxstyleliteralemphasis{\sphinxupquote{str}}) \textendash{} Texto a apresentar ao utilizador para solicitar a entrada

\sphinxlineitem{Retorno}
\sphinxAtStartPar
Lista de números decimais inseridos pelo utilizador

\sphinxlineitem{Tipo de retorno}
\sphinxAtStartPar
list{[}float{]}

\sphinxlineitem{Levanta}
\sphinxAtStartPar
\sphinxstyleliteralstrong{\sphinxupquote{ValueError}} \textendash{} Se algum dos valores inseridos não puder ser
convertido para float

\end{description}\end{quote}

\sphinxAtStartPar
Por exemplo:

\begin{sphinxVerbatim}[commandchars=\\\{\}]
\PYG{g+gp}{\PYGZgt{}\PYGZgt{}\PYGZgt{} }\PYG{n}{pedir\PYGZus{}lista\PYGZus{}floats}\PYG{p}{(}\PYG{l+s+s2}{\PYGZdq{}}\PYG{l+s+s2}{Digite valores: }\PYG{l+s+s2}{\PYGZdq{}}\PYG{p}{)}
\PYG{g+go}{Digite valores: 1.5,2.7,3.9}
\PYG{g+go}{[1.5, 2.7, 3.9]}
\end{sphinxVerbatim}

\end{fulllineitems}

\index{pedir\_lista\_inteiros() (no módulo main)@\spxentry{pedir\_lista\_inteiros()}\spxextra{no módulo main}}

\begin{fulllineitems}
\phantomsection\label{\detokenize{main:main.pedir_lista_inteiros}}
\pysigstartsignatures
\pysiglinewithargsret
{\sphinxcode{\sphinxupquote{main.}}\sphinxbfcode{\sphinxupquote{pedir\_lista\_inteiros}}}
{\sphinxparam{\DUrole{n}{mensagem}}}
{}
\pysigstopsignatures
\sphinxAtStartPar
Solicita ao utilizador uma lista de números inteiros.

\sphinxAtStartPar
A função apresenta uma mensagem ao utilizador e espera que este
insira uma sequência de números separados por vírgula. Cada valor
é convertido para int.
\begin{quote}\begin{description}
\sphinxlineitem{Parâmetros}
\sphinxAtStartPar
\sphinxstyleliteralstrong{\sphinxupquote{mensagem}} (\sphinxstyleliteralemphasis{\sphinxupquote{str}}) \textendash{} Texto a apresentar ao utilizador para solicitar a entrada

\sphinxlineitem{Retorno}
\sphinxAtStartPar
Lista de números inteiros inseridos pelo utilizador

\sphinxlineitem{Tipo de retorno}
\sphinxAtStartPar
list{[}int{]}

\sphinxlineitem{Levanta}
\sphinxAtStartPar
\sphinxstyleliteralstrong{\sphinxupquote{ValueError}} \textendash{} Se algum dos valores inseridos não puder ser
convertido para int

\end{description}\end{quote}

\sphinxAtStartPar
Por exemplo:

\begin{sphinxVerbatim}[commandchars=\\\{\}]
\PYG{g+gp}{\PYGZgt{}\PYGZgt{}\PYGZgt{} }\PYG{n}{pedir\PYGZus{}lista\PYGZus{}inteiros}\PYG{p}{(}\PYG{l+s+s2}{\PYGZdq{}}\PYG{l+s+s2}{Digite valores: }\PYG{l+s+s2}{\PYGZdq{}}\PYG{p}{)}
\PYG{g+go}{Digite valores: 1,2,3}
\PYG{g+go}{[1, 2, 3]}
\end{sphinxVerbatim}

\end{fulllineitems}


\sphinxstepscope


\chapter{modulo\_lista module}
\label{\detokenize{modulo_lista:module-modulo_lista}}\label{\detokenize{modulo_lista:modulo-lista-module}}\label{\detokenize{modulo_lista::doc}}\index{module@\spxentry{module}!modulo\_lista@\spxentry{modulo\_lista}}\index{modulo\_lista@\spxentry{modulo\_lista}!module@\spxentry{module}}\index{agrupar\_por\_tamanho() (no módulo modulo\_lista)@\spxentry{agrupar\_por\_tamanho()}\spxextra{no módulo modulo\_lista}}

\begin{fulllineitems}
\phantomsection\label{\detokenize{modulo_lista:modulo_lista.agrupar_por_tamanho}}
\pysigstartsignatures
\pysiglinewithargsret
{\sphinxcode{\sphinxupquote{modulo\_lista.}}\sphinxbfcode{\sphinxupquote{agrupar\_por\_tamanho}}}
{\sphinxparam{\DUrole{n}{lista\_strings}}}
{}
\pysigstopsignatures
\end{fulllineitems}

\index{calcular\_frequencia() (no módulo modulo\_lista)@\spxentry{calcular\_frequencia()}\spxextra{no módulo modulo\_lista}}

\begin{fulllineitems}
\phantomsection\label{\detokenize{modulo_lista:modulo_lista.calcular_frequencia}}
\pysigstartsignatures
\pysiglinewithargsret
{\sphinxcode{\sphinxupquote{modulo\_lista.}}\sphinxbfcode{\sphinxupquote{calcular\_frequencia}}}
{\sphinxparam{\DUrole{n}{lista}}}
{}
\pysigstopsignatures
\end{fulllineitems}

\index{filtrar\_pares() (no módulo modulo\_lista)@\spxentry{filtrar\_pares()}\spxextra{no módulo modulo\_lista}}

\begin{fulllineitems}
\phantomsection\label{\detokenize{modulo_lista:modulo_lista.filtrar_pares}}
\pysigstartsignatures
\pysiglinewithargsret
{\sphinxcode{\sphinxupquote{modulo\_lista.}}\sphinxbfcode{\sphinxupquote{filtrar\_pares}}}
{\sphinxparam{\DUrole{n}{lista\_numeros}}}
{}
\pysigstopsignatures
\end{fulllineitems}


\sphinxstepscope


\chapter{modulo\_matematica module}
\label{\detokenize{modulo_matematica:module-modulo_matematica}}\label{\detokenize{modulo_matematica:modulo-matematica-module}}\label{\detokenize{modulo_matematica::doc}}\index{module@\spxentry{module}!modulo\_matematica@\spxentry{modulo\_matematica}}\index{modulo\_matematica@\spxentry{modulo\_matematica}!module@\spxentry{module}}\index{calcular\_desvio\_padrao() (no módulo modulo\_matematica)@\spxentry{calcular\_desvio\_padrao()}\spxextra{no módulo modulo\_matematica}}

\begin{fulllineitems}
\phantomsection\label{\detokenize{modulo_matematica:modulo_matematica.calcular_desvio_padrao}}
\pysigstartsignatures
\pysiglinewithargsret
{\sphinxcode{\sphinxupquote{modulo\_matematica.}}\sphinxbfcode{\sphinxupquote{calcular\_desvio\_padrao}}}
{\sphinxparam{\DUrole{n}{numeros}}}
{}
\pysigstopsignatures
\end{fulllineitems}

\index{calcular\_media\_ponderada() (no módulo modulo\_matematica)@\spxentry{calcular\_media\_ponderada()}\spxextra{no módulo modulo\_matematica}}

\begin{fulllineitems}
\phantomsection\label{\detokenize{modulo_matematica:modulo_matematica.calcular_media_ponderada}}
\pysigstartsignatures
\pysiglinewithargsret
{\sphinxcode{\sphinxupquote{modulo\_matematica.}}\sphinxbfcode{\sphinxupquote{calcular\_media\_ponderada}}}
{\sphinxparam{\DUrole{n}{notas}}\sphinxparamcomma \sphinxparam{\DUrole{n}{pesos}}}
{}
\pysigstopsignatures
\end{fulllineitems}

\index{converter\_temperatura() (no módulo modulo\_matematica)@\spxentry{converter\_temperatura()}\spxextra{no módulo modulo\_matematica}}

\begin{fulllineitems}
\phantomsection\label{\detokenize{modulo_matematica:modulo_matematica.converter_temperatura}}
\pysigstartsignatures
\pysiglinewithargsret
{\sphinxcode{\sphinxupquote{modulo\_matematica.}}\sphinxbfcode{\sphinxupquote{converter\_temperatura}}}
{\sphinxparam{\DUrole{n}{temperatura}}\sphinxparamcomma \sphinxparam{\DUrole{n}{escala\_origem}}\sphinxparamcomma \sphinxparam{\DUrole{n}{escala\_destino}}}
{}
\pysigstopsignatures
\end{fulllineitems}


\sphinxstepscope


\chapter{modulo\_texto module}
\label{\detokenize{modulo_texto:module-modulo_texto}}\label{\detokenize{modulo_texto:modulo-texto-module}}\label{\detokenize{modulo_texto::doc}}\index{module@\spxentry{module}!modulo\_texto@\spxentry{modulo\_texto}}\index{modulo\_texto@\spxentry{modulo\_texto}!module@\spxentry{module}}\index{contar\_vogais() (no módulo modulo\_texto)@\spxentry{contar\_vogais()}\spxextra{no módulo modulo\_texto}}

\begin{fulllineitems}
\phantomsection\label{\detokenize{modulo_texto:modulo_texto.contar_vogais}}
\pysigstartsignatures
\pysiglinewithargsret
{\sphinxcode{\sphinxupquote{modulo\_texto.}}\sphinxbfcode{\sphinxupquote{contar\_vogais}}}
{\sphinxparam{\DUrole{n}{texto}}}
{}
\pysigstopsignatures
\end{fulllineitems}

\index{criptografar\_cesar() (no módulo modulo\_texto)@\spxentry{criptografar\_cesar()}\spxextra{no módulo modulo\_texto}}

\begin{fulllineitems}
\phantomsection\label{\detokenize{modulo_texto:modulo_texto.criptografar_cesar}}
\pysigstartsignatures
\pysiglinewithargsret
{\sphinxcode{\sphinxupquote{modulo\_texto.}}\sphinxbfcode{\sphinxupquote{criptografar\_cesar}}}
{\sphinxparam{\DUrole{n}{texto}}\sphinxparamcomma \sphinxparam{\DUrole{n}{deslocamento}}}
{}
\pysigstopsignatures
\end{fulllineitems}

\index{gerar\_acronimo() (no módulo modulo\_texto)@\spxentry{gerar\_acronimo()}\spxextra{no módulo modulo\_texto}}

\begin{fulllineitems}
\phantomsection\label{\detokenize{modulo_texto:modulo_texto.gerar_acronimo}}
\pysigstartsignatures
\pysiglinewithargsret
{\sphinxcode{\sphinxupquote{modulo\_texto.}}\sphinxbfcode{\sphinxupquote{gerar\_acronimo}}}
{\sphinxparam{\DUrole{n}{texto}}}
{}
\pysigstopsignatures
\end{fulllineitems}



\renewcommand{\indexname}{Índice de Módulos do Python}
\begin{sphinxtheindex}
\let\bigletter\sphinxstyleindexlettergroup
\bigletter{m}
\item\relax\sphinxstyleindexentry{main}\sphinxstyleindexpageref{main:\detokenize{module-main}}
\item\relax\sphinxstyleindexentry{modulo\_lista}\sphinxstyleindexpageref{modulo_lista:\detokenize{module-modulo_lista}}
\item\relax\sphinxstyleindexentry{modulo\_matematica}\sphinxstyleindexpageref{modulo_matematica:\detokenize{module-modulo_matematica}}
\item\relax\sphinxstyleindexentry{modulo\_texto}\sphinxstyleindexpageref{modulo_texto:\detokenize{module-modulo_texto}}
\end{sphinxtheindex}

\renewcommand{\indexname}{Índice}
\printindex
\end{document}